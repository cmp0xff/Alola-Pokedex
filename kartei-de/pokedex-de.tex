%%
%    * ----------------------------------------------------------------
%    * "THE BEER-WARE LICENSE" (Revision 42/023):
%    * Ronny Bergmann <mail@rbergmann.info> wrote this file. As long as
%    * you retain this notice you can do whatever you want with this
%    * stuff. If we meet some day and you think this stuff is worth it,
%    * you can buy me a beer or a coffee in return.
%    * ----------------------------------------------------------------
%
%
% A german example using the Kartei.cls - including print and toc as
% options, hence all pages are Din A4.
%
% Last Change: Kartei 1.9, 2012/01/04
%
\documentclass[a7paper,10pt,%grid=both
,toc
,print
]{kartei}
\usepackage[utf8]{inputenc} %UTF8
%UTF8
%\usepackage{xltxtra}

%\usepackage{hyperref}

\begin{document}

\setcardpagelayout

% 0001
\begin{karte}[Notdünger]{
\includegraphics{../regular/bulbasaur}
\includegraphics{../shiny/bulbasaur} \\
Saurier; Samen
}[Chlorophyll]
\emph{Bisasam} macht gern einmal ein Nickerchen im Sonnenschein. Auf seinem
Rücken trägt es einen Samen. Indem es Sonnenstrahlen aufsaugt, wird der Samen
zunehmend größer.

\vspace{5pt}
\begin{tabular}{lll}
der Saurier & des Sauriers & die Saurier \\
der Samen & des Samens & die Samen \\
die Not & der Not & die Nöte \\
der Dünger & des Düngers & die Dünger \\
das Chlorophyll & des Chlorophylls &
\end{tabular}
\vspace{5pt}

Pflanze, Gift
\end{karte}


% 0002
\begin{karte}[Notdünger]{
\includegraphics{../regular/ivysaur}
\includegraphics{../shiny/ivysaur} \\
Saurier; Knospe
}[Chlorophyll]
\emph{Bisaknosp} hat eine Knospe auf seinem Rücken. Beine und Rumpf sind kräftig
genug, um sein Gewicht zu tragen. Wenn es lange in der Sonne liegt, ist das ein
Anzeichen dafür, dass die Knospe bald blüht.

\vspace{5pt}
\begin{tabular}{lll}
der Saurier & des Sauriers & die Saurier \\
die Knopse & der Knopse & die Knopsen \\
die Not & der Not & die Nöte \\
der Dünger & des Düngers & die Dünger \\
das Chlorophyll & des Chlorophylls &
\end{tabular}
\vspace{5pt}

Pflanze, Gift
\end{karte}


% 0003
\begin{karte}[Notdünger]{
\includegraphics{../regular/venusaur}
\includegraphics{../shiny/venusaur} \\
Saurier; Flora
}[Chlorophyll; Speckschicht]
\emph{Bisaflor} hat eine Blume auf seinem Rücken. Wenn sie viel Nahrung und
Sonne aufnimmt, verfärbt sie sich bunt. Der Duft der Blume mildert die Emotionen
der Menschen.

\vspace{5pt}
\begin{tabular}{lll}
der Saurier & des Sauriers & die Saurier \\
die Flora & der Flora & die Floren \\
die Not & der Not & die Nöte \\
der Dünger & des Düngers & die Dünger \\
das Chlorophyll & des Chlorophylls & \\
der Speck & des Speck[e]s & \\
die Schicht & der Schicht & die Schichten \\
\end{tabular}
\vspace{5pt}

Pflanze, Gift
\end{karte}


% 0004
\begin{karte}[Großbrand]{
\includegraphics{../regular/charmander}
\includegraphics{../shiny/charmander} \\
Glut; Salamander
}[Solarkraft]
Die Flamme auf seiner Schweifspitze zeigt seine Gefühlslage an. Sie flackert,
wenn \emph{Glumanda} zufrieden ist. Wenn dieses Pokémon wütend wird, lodert die
Flamme gewaltig.

\vspace{5pt}
\begin{tabular}{lll}
die Glut	&	der Glut	&	die Gluten \\
der Salamander	&	der Salamanders	&	die Salamander \\
groß \\
der Brand	&	des Brand[e]s	&	die Brände \\
solar \\
die Kraft	&	der Kraft	&	die Kräfte \\
\end{tabular}
\vspace{5pt}

Feuer
\end{karte}


% 0005
\begin{karte}[Großbrand]{
\includegraphics{../regular/charmeleon}
\includegraphics{../shiny/charmeleon} \\
Glut; Echse
}[Solarkraft]
Gnadenlos besiegt \emph{Glutexo} seine Gegner mit seinen scharfen Klauen. Wenn 
es auf starke Gegner trifft, wird es wütend und die Flamme auf seiner 
Schweifspitze flackert in einem bläulichen Ton.

\vspace{5pt}
\begin{tabular}{lll}
die Glut	&	der Glut	&	die Gluten \\
die Echse	&	der Echse	&	die Echsen \\
groß \\
der Brand	&	des Brand[e]s	&	die Brände \\
solar \\
die Kraft	&	der Kraft	&	die Kräfte \\
\end{tabular}
\vspace{5pt}

Feuer
\end{karte}


% 0006
\begin{karte}[Großbrand; Solarkraft]{
\includegraphics{../regular/charizard}
\includegraphics{../shiny/charizard} \\
Glut; Drache, Rakete
}[Krallenwucht; Dürre]
\emph{Glurak} fliegt durch die Lüfte, um starke Gegner aufzuspüren. Sein heißer 
Feueratem bringt alles zum Schmelzen. Aber es richtet seinen Feueratem nie auf 
schwächere Gegner.

\vspace{5pt}
\begin{tabular}{lll}
die Glut	&	der Glut	&	die Gluten \\
der Drache	&	des Drachen	&	die Drachen \\
die Rakete	&	der Rakete	&	die Raketen	\\
groß \\
der Brand	&	des Brand[e]s	&	die Brände \\
solar \\
die Kraft	&	der Kraft	&	die Kräfte \\
die Kralle	&	der Kralle	&	die Krallen \\
die Wucht	&	der Wucht	&	 \\
die Dürre	&	der Dürre	&	die Dürren
\end{tabular}
\vspace{5pt}

Feuer, Flug; Feuer, Drache
\end{karte}


% 0007
\begin{karte}[Sturzbach]{
\includegraphics{../regular/squirtle}
\includegraphics{../shiny/squirtle} \\
Schildkröte
}[Regengenuss]
\emph{Schiggys} Panzer dient nicht nur zum Schutz. Die runde Form und die
Furchen auf der Oberfläche verringern den Widerstand im Wasser, sodass dieses 
Pokémon sehr schnell schwimmen kann.

\vspace{5pt}
\begin{tabular}{lll}
die Schildkröte	&	der Schildkröte	&	die Schildkröten \\
der Schild	&	des Schild[e]s	&	die Schilde \\
die Kröte	&	der Kröte	&	die Kröten \\
der Sturz	&	des Sturzes	&	die Stürze \\
der Bach	&	des Bach[e]s	&	die Bäche \\
die Kraft	&	der Kraft	&	die Kräfte \\
das Genuss	&	des Genuss	&	die Genera \\
\end{tabular}
\vspace{5pt}

Wasser
\end{karte}


% 0008
\begin{karte}[Sturzbach]{
\includegraphics{../regular/wartortle}
\includegraphics{../shiny/wartortle} \\
Schildkröte; Locke
}[Regengenuss]
\emph{Schillok} hat einen langen, buschigen Schweif, dessen Farbe intensiver 
wird, wenn es altert. Die Kratzer auf seinem Panzer zeugen von seiner 
Kampfkraft.

\vspace{5pt}
\begin{tabular}{lll}
die Schildkröte	&	der Schildkröte	&	die Schildkröten \\
der Schild	&	des Schild[e]s	&	die Schilde \\
die Kröte	&	der Kröte	&	die Kröten \\
die Locke	&	der Locke	&	die Locken \\
der Sturz	&	des Sturzes	&	die Stürze \\
der Bach	&	des Bach[e]s	&	die Bäche \\
die Kraft	&	der Kraft	&	die Kräfte \\
das Genuss	&	des Genuss	&	die Genera \\
\end{tabular}
\vspace{5pt}

Wasser
\end{karte}


% 0009
\begin{karte}[Sturzbach; Regengenuss]{
\includegraphics{../regular/blastoise}
\includegraphics{../shiny/blastoise} \\
Locke
}[Megawumme]
\emph{Turtok} besitzt Wasserdüsen, die aus seinem Panzer herausragen. Diese sind
sehr präzise. Es kann Wassergeschosse so genau verschießen, dass es damit aus
fast 50 m leere Dosen trifft.

\vspace{5pt}
\begin{tabular}{lll}
die Locke	&	der Locke	&	die Locken \\
der Sturz	&	des Sturzes	&	die Stürze \\
der Bach	&	des Bach[e]s	&	die Bäche \\
die Kraft	&	der Kraft	&	die Kräfte \\
das Genuss	&	des Genuss	&	die Genera \\
die Wumme	&	der Wumme	&	die Wummen \\
\end{tabular}
\vspace{5pt}

Wasser
\end{karte}


% 0010
\begin{karte}[Puderabwehr]{
\includegraphics{../regular/caterpie}
\includegraphics{../shiny/caterpie} \\
Raupe
}[Angsthase]
\emph{Raupy} ist sehr gefräßig, es kann Blätter verschlingen, die größer sind
als es selbst. Seine Antennen sondern einen übel riechenden Gestank ab.

\vspace{5pt}
\begin{tabular}{lll}
die Raupe	&	der Raupe	&	die Raupen \\
der Puder	&	des Puders	&	die Puder \\
die Abwehr	&	der Abwehr	&	 \\
der Angsthase	&	des Angsthasen	&	die Angsthasen \\
die Angst	&	der Angst	&	die Ängste \\
der Hase	&	des Hasen	&	die Hasen \\
\end{tabular}
\vspace{5pt}

Käfer
\end{karte}


% 0011
\begin{karte}[Expidermis]{
\includegraphics{../regular/metapod}
\includegraphics{../shiny/metapod} \\
sicher; Kokon
}[]
Der Panzer dieses Pokémon ist hart wie Stahl. \emph{Safcon} bewegt sich kaum, da
es das weiche Innere unter seiner harten Schale auf seine Entwicklung
vorbereitet.

\vspace{5pt}
\begin{tabular}{lll}
der Kokon	&	des Kokons	&	die Kokons \\
die Epidermis	&	der Epidermis	&	die Epidermen \\
\end{tabular}
\vspace{5pt}

Käfer
\end{karte}


% 0012
\begin{karte}[Facettenauge]{
\includegraphics{../regular/butterfree}
\includegraphics{../shiny/butterfree} \\
Schmetterling; Bö
}[Aufwertung]
\emph{Smettbos} größte Fähigkeit ist das Aufspüren köstlichen Blüten\-honigs. Es 
findet sogar Honig in Blumen, die fast 10 km von seinem Nest entfernt blühen.

\vspace{5pt}
\begin{tabular}{lll}
der Schmetterling	&	des Schmetterlings	&	die Schmetterlinge \\
die Bö	&	der Bö	&	die Böen \\
das Facettenauge	&	des Facettenauges	&	die Facettenaugen\\
die Facette	&	der Facette	&	die Facetten \\
das Auge	&	des Auges	&	die Augen \\
\end{tabular}
\vspace{5pt}

Käfer, Flug
\end{karte}

% \begin{karte}[Zahlenkunde]{Was ist der Unterschied in der Verwendung von Drölf 
% und $n$ bei Ihnen?}
% $n$ wird verwendet für Zahlen bis hin zu verdammt groß, Drölf nur bis hin zu 
% verdammt.

% \end{karte}
% %  \section{Informatik}
% %  \subsection{Spaß mit Verweisen}
% %  %Den Kommentar im Stil ändern

% \renewcommand{\kommentarstil}{\textsc}

% \begin{karte}{Was ist verschränkte Rekursion ?}[ein Beispiel für Label]
% \label{karte:antwort} Siehe Karte \ref{karte:frage}
% \end{karte}
% %  \subsection{Spaß mit Verweisen II}
% \begin{karte}{Was ist die Antwort auf Karte \ref{karte:antwort} ?}
% \label{karte:frage}
%   Hier kommt man eigentlich gar nicht hin. Hier gibt es also nichts zu sehen, 
% bitte blättern sie unauffällig weiter.
% \end{karte}

\end{document}